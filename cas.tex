\chapter{Encryption of the DVB}
There are many parts that are needed to provide DVB with a secure way of 
transmitting streams without facing the risk of content getting stolen. The 
most important parts are:

\begin{itemize}
\item Scrambler - explained in chapter \ref{ch:Scrambling}
\item CAS - explained in section \ref{sec:CAS}
\item Common Interface - explained in section \ref{sec:CI}
\item Descrambler - the inverse of a scrambler.
\end{itemize}

\section{CAS} \label{sec:CAS}
Conditional Access is used to make sure that a user fulfills an amount of 
criteria before being able to view content. To make sure that a user that 
demands data does, the CW used to scramble the data is encrypted by a CAS, and 
only users that are allowed to view said material are allowed to. A CAS consists 
of an EMM-generator and an ECM-generator in addition to some other parts. We are 
only interrested in the ECMG and EMMG since they are the parts that affect the 
scrambler. The ECM is generated using the CW, while the EMM is generated based 
on information related to the user. That information varies, but might relate to 
what channels the user is allowed to access and when the subscription ends. A TV 
will not broadcast any channels without receiving an EMM telling it to do so.
\Warning[Source]{You need to find more sources than just Patrik}

%This might just be a the same thing as the information above, but in simpler terms? Might remove this
A simple example is that a user needs to pay for tv-services to be able to 
watch certain channels. The content provider generates an EMM that tells the 
smart-card if the user has paid for the material it requests. The content 
provider also generates an ECM based on the CW, which the smart-card decrypts 
and passes to the descrambler if the EMM allows it. \Warning[Repetition]{Is this merely a repetition?}

CA-modules. What are they, how and where are they connected to the system? I 
think that the CAM is where you input the smart-card. 

\subsection{Standards}
The three most common CAS in Sweden are Viaccess, Conax and Strong. You need to 
buy a specific CAM depending on what distributor you choose for your channels. 
This 

\subsubsection{Viaccess}
Viaccess is the CA-module used by Boxer. This is currently the most commonly used
 distributor in Sweden.
%Boxer supports CI+

\subsubsection{Conax}
Conax is the CA-module used by Com Hem. This is currently the second most used
distributor in Sweden. Conax has got its headquarter in Norway and is a part of the Telenor Group, which deals with mobile telecommunications \citep{Conax}.
%Com Hem supports CI+

\subsubsection{Strong}
Strong is the CA-module used by Canal Digital. This is currently the third most 
used distributor in Sweden.
%Canal Digital does not support CI+

\section{DVB-SimulCrypt} \label{sec:Simul}
DVB-SimulCrypt is widespread in Europe, and works as an interface between the 
head-end and the CAS \citep{SimulCrypt:2008}.

SimulCrypt encourages the use of several CAS at once \citep[p. 17]{SimulCrypt:2008}. 
But how does it do that? 
Why and how can we do that? Is that because the scrambling is the same, where the
only thing differentiating the CAS is the encoding of the control word?

\section{Common Interface} \label{sec:CI}
THE COMMON INTERFACE CAN'T READ, SO IT WON'T MIND IF I WRITE IN ALL-CAPS ;\_\_\_;
IT ALSO CAN'T UNDERSTAND EMOJIS, WHICH MEANS THAT I CAN INPUT AS MANY STUPID 
THINGS AS I WANT TO. stupid common interface..

\section{Set-up} \label{sec:setup}
The TS is scrambled using a key which is called a \emph{control word}. The 
control word is at least changed several times per minute, but it is not 
uncommon to change the CW every 10 seconds. This means that breaking the key 
has very little effect since it will just be changed in a few seconds even if 
you manage to break it. The control word is generated randomly to make sure 
that consecutive control words are not related to each other. 

The control word is scrambled, then sent to the content provider. The content 
provider then encrypts the CW as an \emph{entitlement control message} (ECM). 
The content provider also generates an \emph{entitlement management message} 
(EMM) which tells the smart-card if the user is allowed access to the data. The 
ECM and EMM are then sent back to the scrambler where it is attached to the 
scrambled TS. This package is then sent to the user, where the ECM, EMM and TS 
are separated. The ECM and EMM are sent to the box containing the smart-card 
for processing, and the TS is sent directly to the descrambler where it waits 
for the CW. The TS is then descrambled using the CW and the pure data is then 
displayed to the user.
