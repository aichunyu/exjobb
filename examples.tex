\chapter{Examples}\label{app:examples}
This appendix contains an algebraic example of how to perform an 
inverse scrambling in CBC-mode.

\section*{CBC-mode calculations} \label{sec:CBCcalc}
The ciphertext is obtained through the following equation where 
\newline
C_{0} \text{ is the IV} \newline
\text{XOR is noted with }\oplus. \newline
C_i \text{ is the ciphertext} \newline
P_i \text{ is the plaintext} \newline
E_k \text{ is the encryption algorithm} \newline
D_k \text{ is the decryption algorithm} \newline

\begin{equation}
C_{i} = E_{k}(P_{i} \oplus C_{i-1})
\end{equation}

The inverse of the encryption algorithm E_{k} \text{ is the decryption 
algorithm } D_{k}.

The inverse of an XOR-operation is an XOR-operation.

This gives us:

\begin{equation}
D_{k}(C_{i}) = P_{i} \oplus C_{i-1}
\end{equation}
which gives us
\begin{equation}
P_{i} = D_{k}(C_{i})\oplus C_{i-1}
\end{equation}
