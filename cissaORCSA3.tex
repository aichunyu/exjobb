\chapter{CSA3 and CISSA}
There are currently two scrambling algorithms being assessed with the purpose of 
replacing the currently used CSA1. This is done to assure content security for 
yet another ten years.

CSA3 is a hardware-friendly, software-unfriendly scrambling algorithm chosen by 
the ETSI to replace the currently used CSA. \citep[pp. 6--7]{DVB:2013}

CISSA is meant to be a software-friendly algorithm designed to allow descrambling
on CPU-based units such as computers, smart phones and tablets. CISSA is not 
designed to be hardware-unfriendly in spite of it being designed to be 
software-friendly. \citep[p. 9]{DVB:2013} 

Both of the algorithms are to be implemented in hardware for scrambling of data.
The difference is that CSA3 is to make it hard to descramble the material in 
software. Since both of the algorithms are confidential, it is sadly impossible 
to find out what makes the CSA3 algorithm software-unfriendly, while the CISSA 
algorithm is software-friendly. \Warning[Source]{Om jag får be snällt}

\section{CISSA}
CISSA stands for \emph{Common IPTV Software-oriented Scrambling Algorithm} and 
is designed to be software-friendly. Opposite to the CSA3, CISSA is made to be 
easily descrambled in software, so that CPU-based systems such as computers and 
smart-phones can also implement it.  Although it is software-friendly, it is 
supposed to able to be implemented efficiently on hardware as well as in 
software \citep[p. 9]{DVB:2013}.

CISSA is to use the AES-128 block cipher in CBC-mode with a 16 byte IV with the 
value 0x445642544d4350544145534349535341.
\Warning[Find more]{But I don't know the name of the second block cipher}

\subsection{Software friendly}
What makes this algorithm software friendly?
Is it still possible to make use of it on an FPGA - since FPGAs are so
general?

\section{CSA3}
CSA is to be succeeded by CSA3 which is based on a combination of a 128-bit 
AES block cipher, which is simply called the AES-128, and a confidential block 
cipher called the XRC \citep[p. 8]{DVB:2013}.

\subsection{XRC}
XRC stands for eXtended emulation Resistant Cipher and is a confidential cipher 
used in DVB \citep[p. 8]{DVB:2013}.

\subsection{Hardware friendly}
What makes the CSA3 so hardware friendly?
Is it because it is meant to be a secret standard, only to be delivered
on a chip, to make it as good as impossible to reverse-engineer?

\section{Conclusion}
From what I've seen, both the CISSA and CSA3 implement the AES-128 for scrambling
, combined with a secret cipher. The secret cipher for CSA3 is the XRC cipher, 
and the secret CISSA cipher is yet to be known. CISSA sounds like a great idea 
in my opinion, allowing CPU-based units to descramble data streams without a 
dedicated HW-Chip. While that is good and all, CSA3 is a finished standard, and 
will probably be more easily implemented on an FPGA.
