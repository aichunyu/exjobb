\chapter{CSA3 and CISSA}
There are currently two scrambling standards that are being \Warning[Fill]{This section with some yadayada} looked into, and assessed with the thought of replacing the currently used CSA1. This is done to assure content security for the coming 10 years or some other bullshit.

CSA3 is a hardware-friendly, software-unfriendly scrambling algorithm chosen by 
the ETSI to replace the currently used CSA. \citep[pp. 6--7]{DVB:2013}

CISSA is meant to be a software-friendly algorithm made with the purpose of 
allowing descrambling on CPU-based units such as computers, smart phones and 
tablets. \citep[p. 9]{DVB:2013}

\section{CISSA}
CISSA stands for \emph{Common IPTV Software-oriented Scrambling Algorithm} and 
is designed to be software-friendly. Opposite to the CSA3, CISSA is made to be 
easily descrambled in software, so that CPU-based systems such as computers and 
smart-phones can also implement it.  Although it is software-friendly, it is 
supposed to able to be implemented efficiently on hardware as well as in 
software \citep[p. 9]{DVB:2013}.

CISSA is to use the AES-128 block cipher in CBC-mode with a 16 byte IV with the 
value 0x445642544d4350544145534349535341.
\Warning[Find more]{But I don't know the name of the second block cipher}

%If the packet is a TS-packet, both the header and adaptation field are to be left unscrambled no matter what. As a result of this being a Block Cipher, some data in the TS-packet might go unscrambled since AES-128 only scrambles blocks of 16 bytes. This means that in a worst case scenario blocks of up to 15 bytes of data might be left clear. Each packet is scrambled independently of each other, which allows for random access. \citep[pp. 10--11]{DVB:2013}

%If the packet is a PES-packet we have even more rules. The header of the PES-packet shall consist of less than 184-bytes and left clear. \Warning[More information]{Might need to write more here}

\subsection{Software friendly}
What makes this algorithm software friendly?
Is it still possible to make use of it on an FPGA - since FPGAs are so
general?

\section{CSA3}
CSA is to be succeeded by CSA3 which is based on a combination of a 128-bit 
AES block cipher, which is simply called the AES-128, and a confidential block 
cipher called the XRC \citep[p. 8]{DVB:2013}.

\subsection{XRC}
XRC stands for eXtended emulation Resistant Cipher and is a confidential cipher 
used in DVB \citep[p. 8]{DVB:2013}.

\subsection{Hardware friendly}
What makes the CSA3 so hardware friendly?
Is it because it is meant to be a secret standard, only to be delivered
on a chip, to make it as good as impossible to reverse-engineer?

\section{Conclusion}
From what I've seen, both the CISSA and CSA3 implement the AES-128 for scrambling
, combined with a secret cipher. The secret cipher for CSA3 is the XRC cipher, 
and the secret CISSA cipher is yet to be known. CISSA sounds like a great idea 
in my opinion, allowing CPU-based units to descramble data streams without a 
dedicated HW-Chip. While that is good and all, CSA3 is a finished standard, and 
will probably be more easily implemented on an FPGA.
