\chapter{CISSA and CSA3}
There are currently two scrambling algorithms being assessed as replacements to 
the currently used DVB-CSA. This is done to assure content security for 
yet another ten years.

%%%%%%%%%%%%%%
%What are the pros and cons of having hardware scramblers and software scramblers
%Software uses itself of the date to generate a random key. Hardware uses itself 
%of additive scramblers, and other random output generators to generate the key. 
%%%%%%%%%%%%%%

CISSA is meant to be a hardware-friendly as well as software-friendly algorithm 
designed to allow descrambling to be made on CPU-based units such as computers, 
smart phones and tablets \citep[p. 9]{DVB:2013}.

CSA3 is a hardware-friendly, software-unfriendly scrambling algorithm chosen by 
the ETSI to replace the currently used CSA \citep[pp. 6--7]{DVB:2013}. 
Software-unfriendly means that descrambling is designed so that it is highly 
impractical to perform in software, but easily done in hardware.

Both of the algorithms are to be implemented in hardware for scrambling of data.
The difference is that CSA3 is to make it hard to descramble the material using 
software. Since both of the algorithms are confidential, it is sadly impossible 
to find out what makes the CSA3 algorithm software-unfriendly, while the CISSA 
algorithm is software-friendly. \Warning[Source]{Om jag får be snällt}

\section{CISSA}
CISSA stands for \emph{Common IPTV Software-oriented Scrambling Algorithm} and 
is designed to be software-friendly. Opposite to the CSA3, CISSA is made to be 
easily descrambled in software, so that CPU-based systems such as computers and 
smart-phones can also implement it.  Although it is software-friendly, it is 
supposed to able to be implemented efficiently on hardware as well as in 
software \citep[p. 9]{DVB:2013}.

CISSA is to use the AES-128 block cipher in CBC-mode with a 16 byte IV with the 
value 0x445642544d4350544145534349535341. Each TS packet is to be process 
independetly of other TS packets, but each block of data in the payload depends 
on the previous blocks of data in the same payload. Both the header and 
adaptation field are to be left unscrambled. \citep[p. 11]{DVB:2013}

\subsection{Software friendly}
An FPGA implementation of the CISSA algorithm seems likely to be implementable, 
due to the fact that the scrambling of the content is supposed to be made in 
hardware, regardless as to whether the descrambling is supposed to be made 
either in hardware or software.

While having a scrambling algorithm designed to enable viewing on CPU-based units
opens up the market for more users, it might increase the risk for algorithm 
theft. Since reverse-engineering is possible for software implementations, one 
might find the algorithm for descrambling, as well as scrambling through 
inversion of the algorithm. Knowing the algorithm enables cryptoanalysists to 
search for weaknesses in the algorithm, with the purpose of breaking it.

"A cryptosystem should be secure even if everything about the system, except the 
key, is public knowledge." according to Kerckhoffs's Principle.
\Warning[Source]{No no, not Wikipedia}
This means that the only result of having a descrambling method suited for 
hardware as well as software implementation should possibly only result in some 
free implementations showing up. But it being implemented in software should 
therefore not lead to any problem.
%Is this even a problem? I think WISI utilizes a free implementation of the CSA.

\section{CSA3}
The CSA3 scrambling algorithm is based on a combination of an AES 
(\emph{Advanced Encryption Standard}) block cipher using a 128-bit key, which is 
simply called the AES-128, and a confidential block cipher called the XRC 
\citep[p. 8]{DVB:2013}. XRC stands for eXtended emulation Resistant Cipher and 
is a confidential cipher used in DVB \citep[p. 8]{DVB:2013}.

\subsection{Hardware friendly}
The CSA3 is designed to be hardware-friendly, meaning that descrambling through 
software methods is supposed to be next to impossible. Using a software-hostile 
descrambling algorithm means that reverse-engineering and algorithm theft becomes
hard, if even possible. Even though it would decrease the probability of content 
theft, it closes the door to expansion onto the CPU-based units market, which is 
becoming larger and larger.

\section{Conclusion}
From what I've seen, both CISSA and CSA3 implement the AES-128 for scrambling, 
combined with a secret cipher. The secret cipher for CSA3 is the XRC cipher, and 
the secret CISSA cipher is yet to be known. It is not even sure that CISSA is to 
use a secret Cipher, but might instead use itself of merely the CBC-mode of 
operation for the AES128 cipher, or something. 

CISSA sounds like a great idea in my opinion, allowing CPU-based units to 
descramble data streams without a dedicated HW-Chip. While that is good and all, 
CSA3 is a finished standard, and will probably be more easily implemented on an 
FPGA, while CISSA does not yet seem to be ready for the market. Starting out 
with an AES-128 chiper would provide for a basis to continue development of the 
scrambling, either towards the CISSA or the CSA3 solution, on a later stage.
