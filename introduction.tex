\chapter{Introduction}
WISI Norden AB, previously A2B Electronics, is a Swedish company 
founded in 1997. The company is a developer of headend cable-TV 
distribution systems. WISI Norden develops and designs both hardware and
software, with the purpose of providing Digital TV solutions. 

The purpose of this thesis has been to find a replacement to the 
currently implemented scrambler, located in the head-end solutions. The 
previous scrambler needed to be replaced, since it was designed in 
1994 to last for ten years. The scrambler is used to render the 
digital television streams unreadable if the user does not subscribe to 
the encoded channels.

The task was to evaluate and analyze a few potential scrambling 
algorithms, and then choose which was the most suitable to replace the 
currently implemented algorithm in WISI Norden's devices.

\section{Background}
The formerly used \emph{common scrambling algorithm} (CSA) has due to 
recent progresses in television broadcasting become obsolete. CSA was 
designed to make software descrambling hard, if possible, while making 
hardware descrambling fast. 

There are two suggested replacements of CSA. The first one is named 
after the CSA, and is called CSA3. There already exists an algorithm 
called CSA2 , which is basically the same as CSA, just with a different 
key-length. The second algorithm is the software-friendly descrambling 
algorithm CISSA. Both of them are based on the public Advanced 
Encryption Standard - 128 (commonly known as the AES-128). There are 
three versions of the AES, with varying numbers. The number depicts 
what key-length the AES uses.

WISI Norden asked me to evaluate the replacement algorithms, even 
though the CSA is still used in the DVB world. This was done to make 
sure that there was an alternative to the CSA, when other companies 
would start to switch scrambling methods. WISI Norden has also had some 
requests to implement other scrambling methods from clients.

\section{Problem specification}
The task was to analyze the possible replacements for the common 
scrambling algorithm, and decide which one was the most suitable 
replacement. After choosing an algorithm, that algorithm was to be 
implemented from scratch, making decisions to minimize the hardware 
usage while achieving a suitable frequency. The decisions made were to 
be motivated either through simulations or reference litterature.

There were two proposed replacements to be compared and analyzed 
to find what made one of them software-friendly and the other one 
hardware-friendly. 

\section{Constraints}
The thesis has been limited to implement the scrambing algorithm chosen 
in consent between me and the supervisor at WISI Norden. The 
implementation should be optimized towards hardware usage, while 
achieving a suitable frequency, preferably the one used in the rest of 
the Field-Programmable Gate Array (FPGA).

%% While the entire circuit can be made as a combinatorial circuit, it 
%% will most likely fill up the entire FPGA.

\section{Methodology}
The project was split into a set of tasks, to be performed in the order 
written below. Performing the tasks in this order was done to decrease 
the complexity of the seperate tasks.

\begin{itemize}
\item Litterature study
\item Choosing an algorithm
\item Design and test of entities
\item Implementation
\item Optimization
\end{itemize}

I began the research by studying litterature, to find out what 
cryptography was about. This provided some insight into what the 
strenghts and weaknesses the algorithms actually were. This gave me 
some depth, and I chose to start of with a cipher that was actually 
used in both of the algorithms that I studied. Using the gathered 
background information about how the algorithm worked made design 
and testing of the entities rather easy. I initially designed the lower
level entities, which allowed for easier testing of seperate parts of 
the system. Since I already knew that they functioned properly, due to 
low level testing, it was easier to merge them with other entities, to 
build the system bottom-up. 

%% Hur har exjobbet genomförts? Det här behöver skrivas om och förbättras. 
%% Det är ju hur jag jobbat, men jag vet inte hur vettigt det jag skrev 
%% var.

%JAG HAR INGEN ANING.
